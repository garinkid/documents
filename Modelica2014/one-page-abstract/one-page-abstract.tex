%sample file for Modelica 2011 Abstract page

\documentclass[11pt,a4paper]{article}
\usepackage{graphicx}
% uncomment according to your operating system:
% ------------------------------------------------
\usepackage[latin1]{inputenc}    %% european characters can be used (Windows, old Linux)
%\usepackage[utf8]{inputenc}     %% european characters can be used (Linux)
%\usepackage[applemac]{inputenc} %% european characters can be used (Mac OS)
% ------------------------------------------------
\usepackage[T1]{fontenc}   %% get hyphenation and accented letters right
\usepackage{mathptmx}      %% use fitting times fonts also in formulas
\usepackage{caption}
\usepackage{subcaption}
\usepackage{hyperref}

% do not change these lines:
\pagestyle{empty}                %% no page numbers!
\usepackage[left=35mm, right=35mm, top=15mm, bottom=20mm, noheadfoot]{geometry}
%% please don't change geometry settings!

\newcommand{\imagedirectory}[1]{../images/{#1}}

% begin the document
\begin{document}
\thispagestyle{empty}

\title{\textbf{Modelica Model for the youBot Manipulator}}
\author{Rhama Dwiputra\footnote{rhama.dwiputra,roustiam.chakirov,erwin.prassler\}@h-brs.de}
 \quad Alexey Zakharov\footnote{alexey.zhakarov@gmail.com} \quad Roustiam Chakirov\footnotemark[1] \quad Erwin Prassler\footnotemark[1]\\
Bonn-Rhein-Sieg University of Applied Sciences, Department of Computer Science\\
Grantham-Allee 20, 53757 Sankt Augustin} 
\date{} % <--- leave date empty
\maketitle\thispagestyle{empty} %% <-- you need this for the first page

Models and simulation tools are crucial in robotic research.
Although there have been major improvements in the electronic and mechanical field, robots are still expensive equipments. 
The use of models and simulation tools overcome this problem.
This paper presents the development of the Modelica
model for the youBot manipulator.
The youBot is a mobile manipulator designed to
serve as the reference platform for industry, research and education \cite{Bischoff2011}.
Therefore, the Modelica model of the youBot manipulator is of high importance.
The model was developed with a Modelica library for the manipulator's components which provides modularity, reusability and abstraction (Figure \ref{fig:fig1}).
This approach enables component exchange and component-based experiment of the developed model.

\begin{figure}[htb]
\begin{subfigure}[b]{0.5\textwidth}
\centering
\includegraphics[width=5.5cm, angle=0]{\imagedirectory{AbstractionLayer.png}}
\vspace{0.5cm}
\caption{Abstraction Layer}
\label{fig:divide_and_conquer}
\end{subfigure}		
\begin{subfigure}[b]{0.5\textwidth}
\centering
\includegraphics[width=4cm, angle=0]{\imagedirectory{Manipulator_Vis.png}}
\caption{Visualization}
\label{fig:youBot_vis}
\end{subfigure}	
\caption{The youBot's Manipulator Model}
\label{fig:fig1}
\end{figure}

A model is a representation of the actual system and the benefit of having a model only holds true when the model accuracy is known. 
Simulation can result in wrong conclusion when the researcher forget the limitations and condition under which the simulation is valid \cite{Fritzon2004}.
Therefore, a comparison test with the actual system is performed to evaluate the model accuracy and identify the major components which require further improvement.
The test result shows that the model reflects the actual system within a reasonable deviation. 
For future work, the manipulator model is planned to be to be tested with other Modelica tools (OpenModelica, jModelica) and used for hardware-in-the-loop experiments.
The development or design of other manipulator models is also possible through the reusability of the components library. 
The library is publicly available\footnote{\href{http://www.youbot-store.com}{www.youbot-store.com}} to be used for education and research.


\begin{thebibliography}{00}
\addcontentsline{toc}{chapter}{References}

\bibitem{Bischoff2011} R. Bischoff, U. Huggenberger
, and E. Prassler, \emph{``Kuka youbot - a mobile manipulator for research and education''}, in \emph{IEEE Int. Conf. on Robotics and Automat. (ICRA)}, pp. 1-4, May 2011.

\bibitem{Fritzon2004} P. Fritzson, \emph{Principles of Object-Oriented Modeling and Simulation With Modelica 2.1.} IEEE Press, 2004.

\end{thebibliography}

\end{document}
