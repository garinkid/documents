\documentclass[portrait, a1paper]{baposter}

\definecolor{light-blue}{rgb}{0, 0.85, 1}
\definecolor{dark-blue}{rgb}{0, 0.45 , 0.75}
\definecolor{white}{rgb}{1, 1, 1}

\usepackage[font=small,labelfont=bf]{caption}
\usepackage{float}
\usepackage{wrapfig}
\usepackage{graphicx}
\usepackage{caption}
\usepackage{subcaption}
\usepackage{setspace} 
\usepackage{relsize}
\newcommand{\imagedirectory}[1]{../images/{#1}}
\newcommand{\imagecommon}[1]{../../images/{#1}}

\begin{document}

%%% Setting Background Image %%%%%%%%%%%%%%%%%%%%%%%%%%%%%%%%%%%%%%%%%%%%%%%%%%
\background{
	\begin{tikzpicture}[remember picture,overlay]%
	\draw (current page.north west)+(-2em,2em) node[anchor=north west]
	{\includegraphics[height=1.1\textheight]{background}};
	\end{tikzpicture}
}

	\begin{poster}
	%poster option
	{
		%headershade=shade,
		headerColorOne=dark-blue,
		headerColorTwo=white,
		headerFontColor=white,
		headerfont=\bf,
		headerborder=open,
		headershape=rectangle,
		borderColor=light-blue,
		background=user,
		%bgColorOne=white,
		columns=3,
		eyecatcher=false,
		textborder=rectangle,
		boxColorOne=white,
		 % Background color of the content boxes
	}
	{
    %eye catcher
	}
	%Title
	{
	\Large{Modelica Model for the youBot Manipulator} \vspace{0.1em}
	}
	%Author
	{
	\small{Rhama Dwiputra$^*$, Alexey Zakharov$^+$, Roustiam Chakirov$^*$, Erwin Prassler$^*$\\
	\scriptsize{
$^*${\{rhama.dwiputra, roustiam.chakirov, erwin.prassler\}@h-brs.de}, $^+${alexey.zhakarov@gmail.com}
		}
		}
	}
	%University Logo
	{
	\includegraphics[width=0.25\textwidth]{\imagecommon{logo-hbrs}}
	}

%---------------------------------------
% Background
%---------------------------------------

	\headerbox{Background}{name=background, column=0, row=0}{
		The \emph{youBot} is a mobile manipulator designed to serve as the reference platform for industry, research and education \cite{Bischoff2011}.
		Due to its frequent use for educational purpose or investigation of new methods, a model of the youBot is highly advantageous.  
	}
%---------------------------------------
% Approach
%---------------------------------------
	
	\headerbox{Approach}{name=Approach, column=1, row=0, span=2}{
\begin{wrapfigure}{r}{0.44\textwidth}
   \begin{center}
   \vspace{-5pt}
	\includegraphics[width=0.44\textwidth]{\imagedirectory{AbstractionLayer}}\\
 	\label{fig:abstraction-layers}
 	 \vspace{5pt}
 	 \smaller{Component-based development.}
 	 \end{center}
 \end{wrapfigure}	
	
The library is developed with emphasize on modularity, and re-usability.
Additionally, the manipulator model is developed in abstraction layers where the lower layer provides a more detailed information and the upper layer provides the general overview of the system.

}



%---------------------------------------
% Result and Evaluation
%---------------------------------------

	\headerbox{Result and Evaluation}{name=resultandevaluation, column=0, below=background, span=3}{
\begin{figure}[H]
   \vspace{-5pt}
	\begin{subfigure}[b]{0.72\textwidth}
	\includegraphics[width=1.0\textwidth]{\imagedirectory{overview}}
	\label{fig:youBot-overview}	
	\end{subfigure}
	~
	\begin{subfigure}[b]{0.24\textwidth}
	\includegraphics[width=1.0\textwidth]{\imagedirectory{parameters-and-evaluation}}
	\label{fig:youBot-overview}	
	\end{subfigure}\\
	\centering   
	\smaller{Model overview, parameter configuration and evaluation}
\end{figure}
 \vspace{-10pt}
A model is a representation of the actual system and the benefit of having a model only holds true when the model is accurate \cite{Fritzon2004}.
the development of the manipulator model is followed by a test with the actual system.


	
	}

%---------------------------------------
% Future work
%---------------------------------------

	\headerbox{Future Work}{name=FutureWork, column=0, below=resultandevaluation, span=3}{
Possible improvements for the developed Modelica library is the development of a more accurate motor model and a more comprehensive evaluation of the manipulator component (the controller components, the power consumption and the dynamic properties).
The manipulator model is planned to be tested with other Modelica tools (OpenModelica, jModelica) and used for hardware-in-the-loop experiments.

	}

%---------------------------------------
% Acknowledgment
%---------------------------------------	
	
	\headerbox{Acknowledgement}{name=Acknowledgement, column=0, below=FutureWork}{
We gratefully acknowledge the continued support of the Bonn-Rhein-Sieg University of Applied Sciences and its academic staffs for the help and support in this project. 
	}

%---------------------------------------
% Reference
%---------------------------------------	
	
	\headerbox{Reference}{name=Reference, column=1, below=FutureWork, span=2}{
	\smaller{
\vspace{-0.4em} 										% Save some space at the beginning
\bibliographystyle{plain}							% Use plain style
\renewcommand{\section}[2]{\vskip 0.05em}		% Omit "References" title

\begin{thebibliography}{1}							% Simple bibliography with widest label of 1
\itemsep=-0.01em										% Save space between the separation
\setlength{\baselineskip}{0.4em}					% Save space with longer lines
\bibitem{Bischoff2011} R. Bischoff, U. Huggenberger, and E. Prassler, \emph{``Kuka youbot - a mobile manipulator for research and education''}, in \emph{IEEE Int. Conf. on Robotics and Automat. (ICRA)}, pp. 1-4, May 2011.
\bibitem{Fritzon2004} P. Fritzson, \emph{Principles of Object-Oriented Modeling and Simulation With Modelica 2.1.} IEEE Press, 2004.
\end{thebibliography}

	}
	}
		
		
		
	\end{poster}
\end{document}