\documentclass[portrait, a1paper]{baposter}

\definecolor{light-blue}{rgb}{0, 0.85, 1}
\definecolor{dark-blue}{rgb}{0, 0.45 , 0.75}
\definecolor{white}{rgb}{1, 1, 1}

\usepackage{wrapfig}
\usepackage{setspace} 
\usepackage{relsize}
\newcommand{\imagedirectory}[1]{../images/{#1}}
\newcommand{\imagecommon}[1]{../../images/{#1}}

\begin{document}
	\begin{poster}
	%poster option
	{
		%headershade=shade,
		headerColorOne=dark-blue,
		headerColorTwo=white,
		headerFontColor=white,
		headerfont=\bf,
		headerborder=open,
		headershape=rectangle,
		borderColor=light-blue,
		background=plain,
		bgColorOne=white,
		columns=3,
		eyecatcher=false,
		textborder=rectangle,
		boxColorOne=white, % Background color of the content boxes
	}
	{
    %eye catcher
	}
	%Title
	{
	\Large{Modelica Model for the youBot Manipulator} \vspace{0.1em}
	}
	%Author
	{
	\small{Rhama Dwiputra$^*$, Alexey Zakharov$^+$, Roustiam Chakirov$^*$, Erwin Prassler$^*$\\
	\scriptsize{
$^*${\{rhama.dwiputra, roustiam.chakirov, erwin.prassler\}@h-brs.de}, $^+${alexey.zhakarov@gmail.com}
		}
		}
	}
	%University Logo
	{
	\includegraphics[width=0.25\textwidth]{\imagecommon{logo-hbrs}}
	}

%---------------------------------------
% Background
%---------------------------------------

	\headerbox{Background}{name=background, column=0, row=0}{
		Models and simulation tools are crucial in robotic research.
		The \emph{youBot} is a mobile manipulator designed to serve as the reference platform for industry, research and education \cite{Bischoff2011}.
		Due to its frequent use as a test subject for educational purpose or investigation of new methods in research institute, a model of the youBot is highly advantageous.  
	}

%---------------------------------------
% Result and Evaluation
%---------------------------------------

	\headerbox{Result and Evaluation}{name=resultandevaluation, column=0, below=background, span=3}{
The youBot Modelica library consists of: 1. the \emph{Controller} package 2. the \emph{Axis} package 3. the \emph{Body} package and 4. the \emph{System} package. \\
	\includegraphics[width=0.5\textwidth]{\imagedirectory{youbot_arm_model_overview}}
	
	}

%---------------------------------------
% Approach
%---------------------------------------
	
	\headerbox{Approach}{name=Approach, column=1, row=0, span=2}{

\begin{wrapfigure}{r}{0.38\textwidth}
   \begin{center}
	\includegraphics[width=0.38\textwidth]{\imagedirectory{AbstractionLayer}}
 \end{center}
 	\caption{The manipulator model is developed with a model library for its components.}
 	\label{fig:abstraction-layers}
 \end{wrapfigure}	
	
The library is developed using the ``divide and conquer'' principle with emphasize on modularity, and re-usability.
The approach enables components exchange and component-based experiment.
Finally, the manipulator model is developed in abstraction layers (Figure \ref{fig:abstraction-layers}) where the lower layer provides a more detailed information in each component and the upper layer provides the general overview of the system.

}

%---------------------------------------
% Future work
%---------------------------------------

	\headerbox{Future Work}{name=FutureWork, column=0, below=resultandevaluation, span=3}{
Possible improvements for the developed Modelica library is the development of a more accurate motor model and a more comprehensive evaluation of the manipulator component (controller components, power consumption and dynamic properties of every rigid body model).
The manipulator model is planned to be tested with other Modelica tools (OpenModelica, jModelica) and used for hardware-in-the-loop experiments.

	}

%---------------------------------------
% Acknowledgment
%---------------------------------------	
	
	\headerbox{Acknowledgement}{name=Acknowledgement, column=0, below=FutureWork}{
We gratefully acknowledge the continued support of the Bonn-Rhein-Sieg University of Applied Sciences for the help and support in this project. 
	}

%---------------------------------------
% Reference
%---------------------------------------	
	
	\headerbox{Reference}{name=Reference, column=1, below=FutureWork, span=2}{
	\smaller{
\vspace{-0.4em} 										% Save some space at the beginning
\bibliographystyle{plain}							% Use plain style
\renewcommand{\section}[2]{\vskip 0.05em}		% Omit "References" title

\begin{thebibliography}{1}							% Simple bibliography with widest label of 1
\itemsep=-0.01em										% Save space between the separation
\setlength{\baselineskip}{0.4em}					% Save space with longer lines
\bibitem{Bischoff2011} R. Bischoff, U. Huggenberger, and E. Prassler, \emph{``Kuka youbot - a mobile manipulator for research and education''}, in \emph{IEEE Int. Conf. on Robotics and Automat. (ICRA)}, pp. 1-4, May 2011.
\bibitem{Fritzon2004} P. Fritzson, \emph{Principles of Object-Oriented Modeling and Simulation With Modelica 2.1.} IEEE Press, 2004.
\end{thebibliography}

	}
	}
		
		
		
	\end{poster}
\end{document}