\documentclass[portrait, a1paper]{baposter}

\definecolor{light-blue}{rgb}{0, 0.85, 1}
\definecolor{dark-blue}{rgb}{0, 0.45 , 0.75}
\definecolor{white}{rgb}{1, 1, 1}


\begin{document}
	\begin{poster}
	%poster option
	{
		%headershade=shade,
		headerColorOne=dark-blue,
		headerColorTwo=white,
		headerFontColor=white,
		headerfont=\bf,
		headerborder=open,
		headershape=rectangle,
		borderColor=light-blue,
		background=plain,
		bgColorOne=white,
		columns=3,
		textborder=rectangle,
		boxColorOne=white, % Background color of the content boxes
	}
	%eye catcher
	{
	}
	%Title
	{
	\LARGE{Modelica Model for the youBot Manipulator}
	}
	%Author
	{
	\large{Rhama Dwiputra, }
	}
	
	\headerbox{Background}{name=background, column=0, row=0}{
		Models and simulation tools are crucial in robotic research.
		The youBot is a mobile manipulator designed to serve as the reference platform for industry, research and education.
		Due to its frequent use as a test subject for educational purpose or investigation of new methods in research institute, a model of the youBot is highly advantageous.  
	}
	
	\headerbox{Result and Evaluation}{name=resultandevaluation, column=0, below=background, span=3}{
The Modelica library for the youBot manipulator in this paper is developed with Dymola.
The library is developed using a ``divide and conquer'' principle with emphasize on modularity, re-usability and abstraction.
This approach enables components exchange and component-based experiment.
Additionally, a template model is provided for components which are frequently used in the manipulator model.
In such cases, the model has adjustable parameter sets to be configured based on its implementation.
Finally, the manipulator model is developed in different abstraction layers (Figure \ref{fig:abstraction-layers}).
The lower layer provides a more detailed information in each component and the upper layer provides the general overview of the system.
	}
	
	\headerbox{Approach}{name=Approach, column=1, row=0, span=2}{
The Modelica library for the youBot manipulator in this paper is developed with Dymola.
The library is developed using a ``divide and conquer'' principle with emphasize on modularity, re-usability and abstraction.
This approach enables components exchange and component-based experiment.
Additionally, a template model is provided for components which are frequently used in the manipulator model.
In such cases, the model has adjustable parameter sets to be configured based on its implementation.
Finally, the manipulator model is developed in different abstraction layers (Figure \ref{fig:abstraction-layers}).
The lower layer provides a more detailed information in each component and the upper layer provides the general overview of the system.
	}
	
	\headerbox{Future Work}{name=FutureWork, column=0, below=resultandevaluation, span=3}{
The Modelica library for the youBot manipulator in this paper is developed with Dymola.
The library is developed using a ``divide and conquer'' principle with emphasize on modularity, re-usability and abstraction.
This approach enables components exchange and component-based experiment.
Additionally, a template model is provided for components which are frequently used in the manipulator model.
In such cases, the model has adjustable parameter sets to be configured based on its implementation.
Finally, the manipulator model is developed in different abstraction layers (Figure \ref{fig:abstraction-layers}).
The lower layer provides a more detailed information in each component and the upper layer provides the general overview of the system.
	}
	
	
	\headerbox{Acknowledgement}{name=Acknowledgement, column=0, below=FutureWork}{
The Modelica library for the youBot manipulator in this paper is developed with Dymola.
The library is developed using a ``divide and conquer'' principle with emphasize on modularity, re-usability and abstraction.
This approach enables components exchange and component-based experiment.
Additionally, a template model is provided for components which are frequently used in the manipulator model.
In such cases, the model has adjustable parameter sets to be configured based on its implementation.
Finally, the manipulator model is developed in different abstraction layers (Figure \ref{fig:abstraction-layers}).
The lower layer provides a more detailed information in each component and the upper layer provides the general overview of the system.
	}
	
	\headerbox{Reference}{name=Reference, column=1, below=FutureWork, span=2}{
The Modelica library for the youBot manipulator in this paper is developed with Dymola.
The library is developed using a ``divide and conquer'' principle with emphasize on modularity, re-usability and abstraction.
This approach enables components exchange and component-based experiment.
Additionally, a template model is provided for components which are frequently used in the manipulator model.
In such cases, the model has adjustable parameter sets to be configured based on its implementation.
Finally, the manipulator model is developed in different abstraction layers (Figure \ref{fig:abstraction-layers}).
The lower layer provides a more detailed information in each component and the upper layer provides the general overview of the system.
	}
		
		
		
	\end{poster}
\end{document}