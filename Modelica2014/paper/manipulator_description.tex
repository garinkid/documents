\section{The youBot Manipulator}
\label{sec:the-youbot-manipulator}
The specification of the manipulator is acquired from the following sources:
\begin{itemize}
\item official youBot website\footnote{\href{www.youbot-store.com}{http://youbot-store.com/
}}, 
\item email communication with the official distributor of the youBot\footnote{\href{info@locomotec.com
}{info@locomotec.com}} and 
\item discussion with researchers from BRICS\footnote{\href{www.best-of-robotics.org}{http://www.best-of-robotics.org/}} who were involved in the development of the youBot's software.
\end{itemize}
This section consist of two subsections, \emph{kinematic chain} and \emph{control system}.
Due to the nature of the robot which is actively being developed, the description presented is subject to changes.


\subsection{Kinematic chain}
\label{ss:kinematic-chain}

The youBot manipulator is a serial chain manipulator with five revolute joints (shown in Figure \ref{fig:youbot-manipulator-real-joints}).
The manipulator is equipped with a two-finger gripper as its end-effector and each finger weights 0.01 kg. 
The fingers' body, position and motion has insignificant influence to the system dynamic when compared to the overall manipulator system.
Therefore, the gripper is modeled only for the visualization purpose.

\begin{figure}[htb]
	\centering
	\includegraphics[height=4.5cm]{\imagedirectory{youbot-manipulator-real-joints.jpg}}
	\caption{The youBot manipulator}
	\label{fig:youbot-manipulator-real-joints}
\end{figure}

The manipulator is 65.5 cm high when fully extended, weights 6.3 kg and has a payload of 0.5 kg.
Each joint is actuated by brushless DC motors and gearboxes with different specifications.
The kinematic chain, joint ranges and dynamic properties of the manipulator are presented in appendix \ref{app:manipulator-specification}.

\subsection{Control System}
\label{ssec:control-system}
The control system accommodates position, velocity and current control in each joint. 
For each joint, the control system consists of: 1. three cascaded \emph{proportional-integral-derivative} or PID controllers, 2. a velocity ramp or v-ramp generator and 3. a space vector pulse width modulation (SVPWM).
Two modes are available for joint position control, \emph{PID} and \emph{v-ramp} mode.
The PID mode calculates the joint velocity in a PID controller whereas in the v-ramp mode, a trapezoidal velocity profile will be generated by the v-ramp generator for the joint velocity.
In this paper, the developed model is based on the joint position control in PID mode.
Figure \ref{fig:controller-overview} shows the overview of the manipulator's controller.
\begin{figure}[htb]
	\centering
	\includegraphics[width=7.6cm]{\imagedirectory{controllerOverview}}
	\caption{Controller overview}
	\label{fig:controller-overview}
\end{figure}

Where $\theta$ is the joint angle, $v$ is the joint velocity and $i$ is the motor current.
The \emph{set} variables ($\theta_{set}$, $v_{set}$, $i_{set}$) are the input values for the PID, the \emph{actual} variables are the values from the manipulator's sensors and the \emph{input} variables are the user defined values.
When controlling the joint position, a user provides the $\theta_{input}$ for the controller and the \emph{Velocity PID} receive the output of the \emph{Position PID} as its $v_{set}$.
When controlling the joint velocity, a user provide the $v_{input}$ for the controller which is directly forwarded as $v_{set}$ to the \emph{Velocity PID} (the output of the \emph{Position PID} in such cases will be ignored).
The \emph{Position PID} is replaced with the v-ramp generator in v-ramp mode.
The PID controllers for position, velocity and current have similar architecture.
As a representative of the PID controllers, Figure \ref{fig:velocity-PID-controller} shows the overview of the PID controller for velocity (\emph{Velocity PID}).
\begin{figure}[htb]
	\centering
	\includegraphics[width=8.2cm]{\imagedirectory{velocityPIDOverview}}
	\caption{Velocity PID overview}
	\label{fig:velocity-PID-controller}
\end{figure}

Where $e$ is the difference between the set value and the actual value. $K_{p}$, $K_ {i}$, and $K_{d}$ are the gain parameters for the controllers.
The output of the \emph{Velocity PID} is forwarded to the \emph{Current PID} as $i_{set}$.
As observed in Figure \ref{fig:velocity-PID-controller}, the Velocity PID controller is similar to the text book PID as follows:
\begin{equation}
\label{eq:textbookpid}
C = K_{p}e(t) + K_{i}\int_{t-\Delta{t}}^{t}e(t)dt + K_{d}\frac{d}{dt}e(t)
\end{equation}
Where $C$ is the controller output and $\Delta t$ is the PID period.
However, the gain parameter in the velocity PID adjusts itself based on the motor velocity as follows:
\begin{equation}\label{eq:gain2-2} 
 k =  \left\{
	 \begin{array}{l l}
    k_{2} & \quad \text{if $\left| v \right| \geq a$}\\
    k_{1} + (\frac{\left| v \right|}{a} * (k_{2} - k_{1}) )  & \quad \text{if $\left| v \right| < a$}
\end{array} \right. 
\end{equation} 
Where $k$ is the gain parameters ($K_{p}$, $K_ {i}$ or $K_{d}$ in Equation \ref{eq:textbookpid}), $k_{1}$ is the lower boundary of the gain parameter, and $k_{2}$ is the upper boundary of the gain parameter value, $v$ is the motor velocity and $a$ is the threshold value for the motor velocity. 
The \emph{Position PID} has the same characteristic as the \emph{Velocity PID}.
Therefore, the \emph{Position PID} and the \emph{Velocity PID} are referred as the non-linear PID.
The non-linear PID enables the user to set different control behaviors for low and high velocity.
Similar to the gain parameters, limiters in the position and velocity controller have non-linear characteristic where the limit value is defined by the motor velocity.

\begin{comment}
In the case of v-ramp mode, the position PID will be disabled and the set velocity is produced by v-ramp generator.
The \emph{textbook} PID controller calculates its output as follows:
\begin{equation}
\label{eq:textbookpid}
C = K_{p}e(t) + K_{i}\int_{t-\Delta{t}}^{t}e(t)dt + K_{d}\frac{d}{dt}e(t)
\end{equation}
\end{comment}


\begin{comment}
\subsubsection{Velocity Ramp Generator}
The velocity ramp or v-ramp generator is used to generate a trapezoidal velocity profile for point-to-point motion.
Figure \ref{fig:vramp-generator-motion-profile} shows the resulting position and acceleration profile from the trapezoidal velocity profile where $s$ is the distance traveled, $v$ is the velocity and $a$ is the acceleration.
\begin{figure}[htb]
	\centering
	\includegraphics[width=8cm]{\imagedirectory{motion_profileVRAMP}}\\
	\caption{V-ramp generator motion profile}
	\label{fig:vramp-generator-motion-profile}
\end{figure}
There are several ways to produce a trapezoidal velocity profile (in theory and practice).
\end{comment}

