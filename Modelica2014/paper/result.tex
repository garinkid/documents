\section{Comparison Test}
\label{sec:result}

A comparison test with the actual system is performed after the development of the manipulator model.
The test purpose is to evaluate the model accuracy and identify the major components which require further development.
The test involves the comparison of the joint position and the joint velocity throughout a point-to-point motion.
For the actual system, the joint velocity is recorded while performing the motion.
The sensor measurement of the joint velocity is assumed to be accurate.
Afterward, the recorded joint velocity is used as input for the System.Dummy model.
In the same setting, the manipulator model ($System.Position$, Figure \ref{fig:system-manipulator-model}) is also performing the same motion.

\begin{figure}[htb]
	\begin{subfigure}[htb]{0.5\textwidth}
	\centering
	\includegraphics[width=6.8cm]{\imagedirectory{EvaluationComparisonVisualNew}}
	\vspace{0.6cm}
	\caption{Error Visualization}
	\vspace{0.4cm}
	\label{fig:error-visualization}
	\end{subfigure}
	\begin{subfigure}[htb]{0.5\textwidth}
	\includegraphics[width=8.2cm]{\imagedirectory{EvaluationErrorJointAngles}}
	\caption{Joint angle difference}
	\label{fig:error-joint-angle}
	\end{subfigure}
	\caption{Test result}
	\label{fig:test-result}
\end{figure}


The manipulator's joints in this test are set to be frictionless.
The motion involves all joints moving 90$^\circ$.
Such motion was chosen so that the resulting error will be the accumulation of the estimated value and approximation in all joints.
Figure \ref{fig:error-visualization} shows the end-effector paths during the motion (the gray-colored youBot manipulator is the starting pose of the motion.) whereas Figure \ref{fig:error-joint-angle} shows the error in joint position.
As expected, the path generated by the model is smoother than that of the actual system as a result of the idealistic conditions in the simulation.
The sum of error from all joint peaked at the value of 0.55 radian.
The error in each joint depends on the maximum velocity parameter ($v_{max}$) in the controller.
As shown in Figure \ref{fig:error-joint-angle}, joint 3 ($v_{max}=4.19$ rad $\cdot$ s$^{-1}$) has a considerably lower peak than joint 5  ($v_{max}=5.90$ rad $\cdot$ s$^{-1} $).
The error in joint angle peaked at two points.
Both peak points happened slightly after the velocity change (from stop to move and slowing down from a constant velocity).
This is consistent with the error in joint velocity as shown in Figure \ref{fig:joint-velocity-comparison}.

\begin{figure}[htb]
	\begin{subfigure}[htb]{0.5\textwidth}
	\centering
	\includegraphics[width=8.2cm]{\imagedirectory{EvaluationJoint3}}
	\vspace{3mm}
	\caption{Joint 3}
	\label{fig:EvaluationJoint3}
	\end{subfigure}		
	\begin{subfigure}[htb]{0.5\textwidth}
	\centering
	\includegraphics[width=8.2cm]{\imagedirectory{EvaluationJoint5}}
	\vspace{3mm}
	\caption{Joint 5}
	\label{fig:EvaluationJoint5}	
	\end{subfigure}
	\caption{Joint velocity comparison}
	\label{fig:joint-velocity-comparison}	
\end{figure}


The joint velocity in the actual system is less stable than in the simulation (Figure \ref{fig:EvaluationJoint3}).
This is the result of the motor vibration which is excluded from the manipulator model.
The ideal motor model results in the deviation on higher velocity (Figure \ref{fig:EvaluationJoint5}) which correspond to the higher error in joint position for joints with higher $v_{max}$ value in its controller.
Similar phenomena in joint velocity and joint position are also found in other joints.
%From this result, it is concluded that the motor model is the major factor of the deviation. 
Other possible contributing aspects in the deviation between the model and the actual system are the inertia tensor estimation, gearbox elasticity/damping, SVPWM approximation and frictionless joints.

