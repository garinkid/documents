\section{The \emph{youBot} Modelica Library}
\label{sec:the_modelica_package_youbot}
The Modelica library for the youBot manipulator in this paper is developed with Dymola.
The library is developed using a ``divide and conquer'' principle with emphasize on modularity, re-usability and abstraction.
This approach enables components exchange and component-based experiment.
Additionally, a template model is provided for components which are frequently used in the manipulator model.
In such cases, the model has adjustable parameter sets to be configured based on its implementation.
Finally, the manipulator model is developed in different abstraction layers (Figure \ref{fig:abstraction-layers}).
The lower layer provides a more detailed information in each component and the upper layer provides the general overview of the system.
%Through inheritance, the selected parameter sets of the component in the lower level can be configured from the upper level.

\begin{figure}[htb]
	\centering
	\includegraphics[height=2.4 cm]{\imagedirectory{AbstractionLayer}}
	\caption{Abstraction layer in a manipulator model}
	\label{fig:abstraction-layers}
\end{figure}

In every modeling process, using estimated values and approximation is unavoidable mainly due to the following reasons:
\begin{itemize}
%\item \emph{Computational load}. 
%The computational load of a model is proportional to the model's complexity. 
%Complex component can be approximated with a simple model under the condition that the model accuracy is consistent.
\item \emph{Limited knowledge}. 
Many parameter set of a dynamic system are estimated through system identification (friction, inertia tensor).
\item \emph{Restricted information}. 
Many manufactures do not provide complete information about their product.
\end{itemize}
The use of estimated values and approximation is presented in the description of each package.
The youBot Modelica library consists of four packages which are:
\begin{itemize} 
	\item \emph{Controller} package, 
	\item \emph{Axis} package, 
	\item \emph{Body} package and 
	\item \emph{System} package.
\end{itemize}
The library is developed with the use of several packages in MSL such as Modelica.Blocks.Math for standard mathematical functions and Modelica.Mechanics for 3-dimensional mechanical systems.
This paper follows the Modelica convention in describing the models.
Model's name or package's name begins with capital letter.
When necessary, the model includes its package name. 
The model $Modelica.Blocks.Interfaces.RealInput$ refers to the model \emph{RealInput} which is inside the package \emph{Interfaces}. 
The Interfaces package is inside the \emph{Blocks} package and the $Blocks$ package is inside the MSL.
An instance of a model is written in lower case aside from a few exceptional cases (e.g. \emph{V} is used for voltage to differentiate from \emph{v} for velocity).
\subsection{\emph{Controller} Package}
\label{ssec:controller_package}
The $Controller$ package consists of the components for the manipulator control system.
The $Controller$ package is divided into three packages which are the \emph{Components} package, the \emph{PIDs} package and the \emph{Modes} package.
The $Controller.Components$ package consists of models which are in the lowest level of abstraction layer.
Figure \ref{fig:model-components} show the models in the \emph{Controller.Component} package. 
\begin{figure}[htb]
	\vspace{0.2cm}
	\begin{subfigure}[htb]{0.11\textwidth}
	\centering
	\includegraphics[height=1.6cm]{\imagedirectory{limiter2.png}}
	\caption{$Limiter2$}
	\label{fig:limiter2}
	\end{subfigure}
	\begin{subfigure}[htb]{0.11\textwidth}
	\centering
	\includegraphics[height=1.6cm]{\imagedirectory{gain2.png}}
	\caption{$Gain2$}
	\label{fig:gain2}
	\end{subfigure}
	\begin{subfigure}[htb]{0.11\textwidth}
	\centering
	\includegraphics[height=1.6cm]{\imagedirectory{DisDer.png}}
	\caption{$DisDer$}
	\label{fig:disder}
	\end{subfigure}
	\begin{subfigure}[htb]{0.11\textwidth}
	\centering
	\includegraphics[height=1.6cm]{\imagedirectory{p2v.png}}
	\caption{$P2V$}
	\label{fig:p2v}
	\end{subfigure}
	\caption{The $Controller.Component$ models}
	\label{fig:model-components}
\end{figure}

Following the Modelica convention, the instance's name of a model is placed on the upper part of the symbols in blue color.
The model \emph{Limiter2} and \emph{Gain2} (Figure \ref{fig:gain2} and \ref{fig:limiter2}) perform the calculation for non-linear PID controller (Equation \ref{eq:gain2-2}).
The model \emph{DisDer} (Figure \ref{fig:disder}) produces the derivative value of a specific time period from a discretized continuous input. The model \emph{P2V} (Figure \ref{fig:p2v}) converts PWM rate to voltage rate. The \emph{P2V} model is an approximation of the SVPWM component in the controller.

The \emph{Controller.PIDs} package consists of three different PID models which are the \emph{Position}, \emph{Velocity} and the \emph{Current} model. As the name suggests, the models are the PID controllers for position, velocity and current in the youBot manipulator (Figure \ref{fig:controller-overview}). 
As a representative, Figure \ref{fig:velocity-PID-modelica-model} shows the $PIDs.Velocity$ model.
\begin{figure}[htb]
	\includegraphics[width=8cm]{\imagedirectory{VelocityPID}}
	\caption{$PIDs.Velocity$}
	\label{fig:velocity-PID-modelica-model}
\end{figure}

Where $v\_set$, $v\_actual$ and $i\_set$ represent $v_{set}$, $v_{actual}$ and $i_{set}$ in Figure \ref{fig:velocity-PID-controller} respectively. 
The additional component $N$ in the model produces the output in mA to mimic the readings of the actual system.

Finally, the $Controller.Modes$ package is for different types of control mode. 
Currently, the available model in the $Modes$ package is the \emph{Position} model.
Figure \ref{fig:position-mode-modelica-model} shows the $Modes.Position$ model.
\begin{figure}[htb]
	\includegraphics[width=8cm]{\imagedirectory{PositionMode}}
	\caption{ $Modes.Position$ }
	\label{fig:position-mode-modelica-model}	
\end{figure}

The $Modes.Position$ model consist of all three PID models from the $Controller.PIDs$ package.
$V\_set$ represent the voltage value which will be connected to the motor's power supply unit.
Using the same approach, the model for other control mode explained in Section \ref{ssec:control-system} can also be developed.

\subsection{\emph{Axis} Package}
\label{ssec:axis_package}
The $Axis$ package consists of the model for joint actuator (motor and control system).
The package is named \emph{Axis} because the model will be connected to the rotating axis of the manipulator's joints. 
The $Axis$ package consists of the \emph{Position} model shown in Figure \ref{fig:position-axis-modelica-model}.
\begin{figure}[htb]
	\includegraphics[width=8.3cm]{\imagedirectory{PositionAxis}}
	\vspace{0.2mm}
	\caption{$Axis.Position$}
	\label{fig:position-axis-modelica-model}
\end{figure}

Where $DCPM$ represents the brushless DC motor model, $PSU$ represents the power supply unit model, $R$ represents the gearbox model, $F$ represents the friction model, $controller$ is the $Modes.Position$ model (Figure \ref{fig:position-mode-modelica-model}), $sens\_v$ represent the joint's velocity sensor, $sens\_theta$ represent the joint's position sensor and $joint$ is the connector to the manipulator's joint model.
The $controller$ output ($V\_set$ in Figure \ref{fig:position-mode-modelica-model}) is connected to $PSU$ and its input is extended for the model input as $theta\_set$.
The output of $sens\_v$, $sens\_theta$ and the value of \emph{DCPM}'s current is connected to the control system ($theta\_actual$, $v\_actual$ and $i\_actual$ in Figure \ref{fig:position-mode-modelica-model}).
Aside from the controller, the component in \emph{Modes.Position} are from MSL.

\subsection{\emph{Body} package}
\label{ssec:body_package}

The $Body$ package consists of models for the rigid body model of the manipulator's kinematic chain.
The $Body$ package has three models which are \emph{Gripper}, \emph{Link} and \emph{Manipulator}.
The $Body.Gripper$ model is the rigid body model of the youBot two finger gripper.
Figure \ref{fig:body-gripper-model} shows the $Body.Gripper$ model.
\begin{figure}[htb]
	\centering
	\includegraphics[width=8.2cm]{\imagedirectory{gripper}}
	\vspace{0.2cm}
	\caption{$Body.Gripper$}
	\label{fig:body-gripper-model}
\end{figure}

Where $RF\_rot$ is the gripper's reference frame rotation, $frame\_a$ is the connector to the previous link model, and $gripper$, $left\_finger$ and $right\_finger$  are the rigid body model of the gripper body, left finger and right finger respectively.
$marker$ is a weightless body model to visualize the path of the manipulator's end effector in simulation and $RF\_vis$ provides the visualization of its reference frame.

The $Body.Link$ model is the rigid body model of a manipulator link.  
Figure \ref{fig:link-model} shows the $Body.Link$ model.
\begin{figure}[htb]
	\centering
	\includegraphics[width=7.8cm]{\imagedirectory{Link}}
	\vspace{0.2mm}
	\caption{$Body.Link$}
	\label{fig:link-model}
\end{figure}

Where $joint$ represent the link's joint which is connected to the actuator model through the connector $motor$, $frame\_b$ is the connector to the next link model and the $body$ represent the rigid body of the link.
$frame\_a$, $RF\_rot$ and $RF\_vis$ represent the same components as in $Body.Gripper$ model.

The $Body.Manipulator$ model represent the rigid body model of the youBot manipulator's kinematic link.
Figure \ref{fig:body-manipulator-model} shows the $Body.Manipulator$ model.
\begin{figure}[htb]
	\centering
	\includegraphics[width=7.5cm]{\imagedirectory{Manipulator}}
	\caption{$Body.Manipulator$}
	\label{fig:body-manipulator-model}
\end{figure}

Where $link1$ represent the first link of the manipulator ($Body.Link$, Figure \ref{fig:link-model}), $gripper$ is the manipulator's gripper model ($Body.Gripper$, Figure \ref{fig:body-gripper-model}), $base$ represent the rigid body of the manipulator's base. The component $base\_pos$ is for defining the manipulator position in the world reference frame. 
The $Body.Manipulator$ model has five connectors ($axis1$ to $axis5$) for each joint model and one connector ($world\_rf$) for the world reference frame.



\subsection{\emph{System} package}
\label{ssec:system_package}
The $System$ package consists of the manipulator ready-to-use models.
The $System$ package has two models which are the \emph{Position} model and the \emph{Dummy} model.
The $System.Position$ model is the rigid body model of youBot manipulator's kinematic chain and its actuators.
Figure \ref{fig:modelica-model} shows the $System.Position$ model and its visualization in Dymola whereas Figure \ref{fig:parameter-adjustment} shows the parameter set configuration for the velocity controller in the manipulator's fifth joint.

\begin{figure}[htb]
	\begin{subfigure}[htb]{0.5\textwidth}
	\includegraphics[width=6.5cm]{\imagedirectory{ManipulatorSystem}}
	\caption{$System.Position$}
	\vspace{6mm}
	\label{fig:system-manipulator-model}
	\end{subfigure}		
	\begin{subfigure}[htb]{0.5\textwidth}
	\centering
	\includegraphics[width=5.7cm]{\imagedirectory{Manipulator_Vis}}
	\caption{Model visualization}
	\label{fig:Manipulator_Vis}
	\end{subfigure}		
	\caption{The youBot manipulator model}
	\label{fig:modelica-model}	
\end{figure}

\begin{figure}[htb]
	\centering
	\includegraphics[width=8.1cm]{\imagedirectory{axis_5_parameters}}
	\caption{Parameters configuration}
	\label{fig:parameter-adjustment}	
\end{figure}

In Figure \ref{fig:system-manipulator-model}, $KL$ represents the rigid body model of the youBot manipulator's kinematic chain, $theta\_5$ represents the user defined joint angle and $axis\_5$ represents the actuator (Figure \ref{fig:position-axis-modelica-model}) for joint 5.
The parameter names in Figure \ref{fig:parameter-adjustment} are consistent with the existing driver and firmware.
The $System.Dummy$ model is the rigid body model of the youBot manipulator (\emph{Body.Manipulator}, Figure \ref{fig:body-manipulator-model}) connected to dummy actuators ($Modelica.Mechanics.Rotational.Speed$).
The user can set the velocity of each joint directly in the $System.Dummy$ model.
The $System.Dummy$ model is used for comparison test in Section \ref{sec:result}.




