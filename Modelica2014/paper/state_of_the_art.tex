\section{State of The Art}
\label{sec:state_of_the_art}

Modelica has been used for modeling spider robotic arm \cite{Ferreti2003}, 6-axis industrial robots \cite{Thuemmel2001, Kazi2002, Hast2009}, 3 DOFs parallel Gantry-Tau robot \cite{Dressler2009}, 5 DOFs manipulator \cite{Chen2009} and mobile platforms \cite{Akesson2009, Pohlmann2012}.
In most cases, a robot model in Modelica is used for investigating the  manipulator's motion control especially in the domain of optimization and system dynamics.
Such research requires the repetition of motions and adjustments to the controller which can have damaging effect when being executed on a real robot.

\cite{Kazi2002} performed optimization through iteration to find a compromise between acceleration, velocity and energy consumption and \cite{Hast2009} solved the minimum time optimization problem for an industrial robot.
\cite{Thuemmel2001} derives the inverse dynamic model of a manipulator using algorithms for differential-algebraic equation available in the Dymola\footnote{\href{http://www.3ds.com/products-services/catia/portfolio/dymola}{www.3ds.com/products-services/catia/portfolio/dymola}} software. 
Dymola was also used in \cite{Chen2009} to design a picking manipulator for agriculture purposes.
\cite{Dressler2009} develops method for kinematic calibration with the Modelica model of parallel Gantry-Tau robot.
Aside in the field of motion control, Modelica robot models have also been used for tele-manipulation \cite{Ferreti2003}, robot communication \cite{Pohlmann2012} and teaching tools \cite{Akesson2009}.

As shown from the work presented in this section, there is a wide range of research with robot models in Modelica.
The Modelica model of the youBot manipulator will enable such research to be performed.
Since the youBot is designed to be the reference platform for academic institute, a Modelica model of the youBot manipulator is of high importance.
